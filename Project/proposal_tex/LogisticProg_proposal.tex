%%%%%%%%%%%%%%%%%%%%%%%%%%%%%%%%%%%%%%%%%
% Short Sectioned Assignment
% LaTeX Template
% Version 1.0 (5/5/12)
%
% This template has been downloaded from:
% http://www.LaTeXTemplates.com
%
% Original author:
% Frits Wenneker (http://www.howtotex.com)
%
% License:
% CC BY-NC-SA 3.0 (http://creativecommons.org/licenses/by-nc-sa/3.0/)
%
%%%%%%%%%%%%%%%%%%%%%%%%%%%%%%%%%%%%%%%%%

%----------------------------------------------------------------------------------------
%	PACKAGES AND OTHER DOCUMENT CONFIGURATIONS
%----------------------------------------------------------------------------------------

\documentclass[paper=a4, fontsize=11pt]{scrartcl} % A4 paper and 11pt font size

\usepackage[T1]{fontenc} % Use 8-bit encoding that has 256 glyphs
\usepackage{fourier} % Use the Adobe Utopia font for the document - comment this line to return to the LaTeX default
\usepackage[english]{babel} % English language/hyphenation
\usepackage{amsmath,amsfonts,amsthm} % Math packages
\usepackage{color} % This is used for highlighting text whenever you need to come back to something
\usepackage{lipsum} % Used for inserting dummy 'Lorem ipsum' text into the template
\usepackage{sectsty} % Allows customizing section commands
\allsectionsfont{\centering \normalfont\scshape} % Make all sections centered, the default font and small caps
\usepackage{fancyhdr} % Custom headers and footers
\pagestyle{fancyplain} % Makes all pages in the document conform to the custom headers and footers
\fancyhead{} % No page header - if you want one, create it in the same way as the footers below
\fancyfoot[L]{} % Empty left footer
\fancyfoot[C]{} % Empty center footer
\fancyfoot[R]{\thepage} % Page numbering for right footer
\renewcommand{\headrulewidth}{0pt} % Remove header underlines
\renewcommand{\footrulewidth}{0pt} % Remove footer underlines
\setlength{\headheight}{13.6pt} % Customize the height of the header

%numberwithin{equation}{section} % Number equations within sections (i.e. 1.1, 1.2, 2.1, 2.2 instead of 1, 2, 3, 4)
%\numberwithin{figure}{section} % Number figures within sections (i.e. 1.1, 1.2, 2.1, 2.2 instead of 1, 2, 3, 4)
%\numberwithin{table}{section} % Number tables within sections (i.e. 1.1, 1.2, 2.1, 2.2 instead of 1, 2, 3, 4)

%\setlength\parindent{0pt} % Removes all indentation from paragraphs - comment this line for an assignment with lots of text

%----------------------------------------------------------------------------------------
%	TITLE SECTION
%----------------------------------------------------------------------------------------

\newcommand{\horrule}[1]{\rule{\linewidth}{#1}} % Create horizontal rule command with 1 argument of height

\title{	
\normalfont \normalsize 
\textsc{Columbia University -- CUNY} \\ [25pt] % Your university, school and/or department name(s)
\horrule{0.5pt} \\[0.4cm] % Thin top horizontal rule
\huge Multimodal Sentiment Analysis \\ % The assignment title
\horrule{2pt} \\[0.5cm] % Thick bottom horizontal rule
}

\author{Logistic Progression -- Jessica Ouyang and Joe Ellis} % Your name

\date{\normalsize\today} % Today's date or a custom date

\begin{document}

\maketitle % Print the title

%---
% Overview 
%---

\section{Overview}

% Task

\subsection{Task}
In this project we propose to take a multi-modal approach to sentiment analysis, and create the capability to extract sentiment from a variety of differnet source, including videos, pictures, and text.
Sentiment Analysis is a widely studied area of text analysis \cite{Pang}, but recently some work has been completed on visual sentiment analysis, such as SentiBank \cite{MM13:sentibank_long}.
We propose to fuse information from visual, audio, and text signals to achieve a more complete representation of sentiment, and more accurately classify sentiment within a variety of sources. 
We propose to also study the difference in sentiment between varying mediums.
For example, the change in sentiment on a subject between Twitter, Youtube, and Broadcast Video News may change drastically.
We propose to explore these differences and propose new models for multi-modal sentiment analysis.

% Data Used

\subsection{Data Used}
We will use a variety of on-line data sources for this project, as well as manually downloaded cable and broadcast news stories. 
We propose to utilize social media sites such as Twitter, Facebook, and Instagram, to gain real-time text and image data that can be processed for sentiment.
We also have access to the past years worth of on-line news articles and news stories through the NewsRover project \cite{MM13:structurednews}.

% Ideas on Techniques
\subsection{Ideas on Techniques}
{\color{red} Need to add this point for the next section.}
{\color{blue} Add some stuff on socially aware content analysis of video/audio content.}

%Groundbreaking
\subsection{Why is it cool?}
This work builds on a very popular portion of research in a way that has not currently been explored. 
Multimodal analysis has shown promise in a variety of fields and sensor fusion techniques have become widely used.
As we move more toward high-bandwidth data sources such as video and audio content, much of the sentiment that we create will be tied up in mediums other than text.
Therefore, the lucrative field of sentiment analysis would benefit from the creation of a framework for multimodal data processing.

%NLP portion
\section{Subject Contributions}
\subsection{NLP Contribution}
This core of this project will be sentiment analysis, and the medium in which sentiment analysis is the most thoroughly developed is in text.
The NLP novelty within this project is the ability to automatically combine text information from multiple different text sources (Twitter, social media, on-line news articles, tv transcripts).
Combining these sources in interesting ways could add a novel portion to the typical NLP processing pipeline.

%ML Portion
\subsection{ML Contributions}
The core of this portion of the contribution is the fusion of features extracted from different modalities.
These features are calculated from different spaces, and therefore can not be easily combined.
Therefore, we look to find intelligent ways to fuse these different features, and this should be a contribution to the ML community.

% Web Technologies
\subsection{Web Technologies}
We plan to create a program that automatically analyses content from multiple sources of available on-line web data.
These sources could include, but are not limited to, Youtube, Twitter, Facebook, Instagram, and Pinterest.
We hope to create programs to process the public stream content that arrives from the accounts.

% Class Member Contributions
\section{Member Contributions}

\subsection{Joe Ellis}
{\color{red} Need to add after we discuss with Jessica}

\subsection{Jessica Ouyang}
{\color{red} Need to add after we discuss with Jessica}




\bibliographystyle{plain}
\bibliography{LogisticProg_proposal}

\end{document}