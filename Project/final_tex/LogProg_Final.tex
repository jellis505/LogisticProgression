%%%%%%%%%%%%%%%%%%%%%%%%%%%%%%%%%%%%%%%%%
% Journal Article
% LaTeX Template
% Version 1.3 (9/9/13)
%
% This template has been downloaded from:
% http://www.LaTeXTemplates.com
%
% Original author:
% Frits Wenneker (http://www.howtotex.com)
%
% License:
% CC BY-NC-SA 3.0 (http://creativecommons.org/licenses/by-nc-sa/3.0/)
%
%%%%%%%%%%%%%%%%%%%%%%%%%%%%%%%%%%%%%%%%%

%----------------------------------------------------------------------------------------
%	PACKAGES AND OTHER DOCUMENT CONFIGURATIONS
%----------------------------------------------------------------------------------------

\documentclass[twoside]{article}

\usepackage{lipsum} % Package to generate dummy text throughout this template

\usepackage[sc]{mathpazo} % Use the Palatino font
\usepackage[T1]{fontenc} % Use 8-bit encoding that has 256 glyphs
\linespread{1.05} % Line spacing - Palatino needs more space between lines
\usepackage{microtype} % Slightly tweak font spacing for aesthetics

\usepackage[hmarginratio=1:1,top=32mm,columnsep=20pt]{geometry} % Document margins
\usepackage{multicol} % Used for the two-column layout of the document
\usepackage[hang, small,labelfont=bf,up,textfont=it,up]{caption} % Custom captions under/above floats in tables or figures
\usepackage{booktabs} % Horizontal rules in tables
\usepackage{float} % Required for tables and figures in the multi-column environment - they need to be placed in specific locations with the [H] (e.g. \begin{table}[H])
\usepackage{hyperref} % For hyperlinks in the PDF

\usepackage{lettrine} % The lettrine is the first enlarged letter at the beginning of the text
\usepackage{paralist} % Used for the compactitem environment which makes bullet points with less space between them

\usepackage{abstract} % Allows abstract customization
\renewcommand{\abstractnamefont}{\normalfont\bfseries} % Set the "Abstract" text to bold
\renewcommand{\abstracttextfont}{\normalfont\small\itshape} % Set the abstract itself to small italic text

\usepackage{titlesec} % Allows customization of titles
\renewcommand\thesection{\Roman{section}} % Roman numerals for the sections
\renewcommand\thesubsection{\Roman{subsection}} % Roman numerals for subsections
\titleformat{\section}[block]{\large\scshape\centering}{\thesection.}{1em}{} % Change the look of the section titles
\titleformat{\subsection}[block]{\large}{\thesubsection.}{1em}{} % Change the look of the section titles

\usepackage{fancyhdr} % Headers and footers
\pagestyle{fancy} % All pages have headers and footers
\fancyhead{} % Blank out the default header
\fancyfoot{} % Blank out the default footer
\fancyhead[C]{Logistic Progression $\bullet$ December 23 $\bullet$ NLP/ML/Web} % Custom header text
\fancyfoot[RO,LE]{\thepage} % Custom footer text

%----------------------------------------------------------------------------------------
%	TITLE SECTION
%----------------------------------------------------------------------------------------

\title{\vspace{-15mm}\fontsize{24pt}{10pt}\selectfont\textbf{Multimodal Sentiment Analysis in YouTube Videos}} % Article title

\author{
\large
\textsc{Joe Ellis and Jessica Ouyang} \\ %\thanks{A thank you or further information}\\[2mm] % Your name 
\normalsize Columbia University \\ % Your institution
\normalsize \href{mailto:jge2105@columbia.edu}{jge2105@columbia.edu}, % Your email address
\normalsize \href{mailto:ouyangj@cs.columbia.edu}{ouyangj@cs.columbia.edu} % Your email address
\vspace{-5mm}
}
\date{}

%----------------------------------------------------------------------------------------

\begin{document}

\maketitle % Insert title

\thispagestyle{fancy} % All pages have headers and footers

%----------------------------------------------------------------------------------------
%	ABSTRACT
%----------------------------------------------------------------------------------------

\begin{abstract}
Jessica
\end{abstract}

%----------------------------------------------------------------------------------------
%	ARTICLE CONTENTS
%----------------------------------------------------------------------------------------

\begin{multicols}{2} % Two-column layout throughout the main article text

\section{Introduction}
Joe Ellis

%----------------------------------------------
% --- Dataset
%------------------------------------------------

\section{DataSet}

\subsection{Data Collection}
Joe 

\subsection{Description of Videos}
Joe 

%Maecenas sed ultricies felis. Sed imperdiet dictum arcu a egestas. 
%\begin{compactitem}
%\item Donec dolor arcu, rutrum id molestie in, viverra sed diam
%\item Curabitur feugiat
%\end{compactitem}
%\lipsum[4] % Dummy text

%------------------------------------------------
% ----- FEATURE EXTRACTION
% ----------------------------------------------


\section{Feature Extraction}
Jessica

\subsection{Text Features}
Jessica

\subsection{Visual Feature}
Joe

\subsection{Audio Features}
Joe

%------------------------------------------------
% ---- Classification
%------------------------------------------------

\section{Classification}
Jessica

%------------------------------------------------
% ---- Experiments
%------------------------------------------------

\section{Experiments}
Joe

\subsection{Single Classifier}
Joe and Jessica

\subsection{Co-Training Classification}
Joe and Jessica

%------------------------------------------------
% ---- Discussion
%------------------------------------------------

\section{Discussion}
Joe or Jessica


%----------------------------------------------------------------------------------------
%	REFERENCE LIST
%----------------------------------------------------------------------------------------

% Add bibtex file here

%----------------------------------------------------------------------------------------

\end{multicols}

\end{document}
